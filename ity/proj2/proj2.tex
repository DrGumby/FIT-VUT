\documentclass[twocolumn,11pt,a4paper]{article}
\usepackage[utf8]{inputenc}
\usepackage[IL2]{fontenc}
\usepackage[left=1.5cm, text={18cm, 25cm}, top=2.5cm]{geometry}
\usepackage[czech]{babel}
\usepackage{hyperref}
\usepackage{times}
\usepackage{amsthm}
\usepackage{amsfonts}
\usepackage[]{amsmath}

\DeclareMathOperator{\vdashop}{\vdash}


\newtheorem{def1}{Definice}

\newtheorem{th1}{Věta}

\title{Typografie a publikování, projekt 2}
\author{Kamil Vojanec (xvojan00)}
\date{}

\begin{document}
\begin{titlepage}
\begin{center}
\Huge
\textsc{Fakulta informačních technologií \\Vysoké učení technické v Brně}\\
\vspace{\stretch{0.382}}
\LARGE Typografie a publikování -- 2. projekt\\
Sazba dokumentů a matematických výrazů
\vfill
\end{center}
{\LARGE 2018 \hfill
Kamil Vojanec (xvojan00)}

\end{titlepage}

\section*{Úvod}
V této úloze si vyzkoušíme sazbu titulní strany, matematických vzorců, prostředí a dalších textových struktur obvyklých pro technicky zaměřené texty (například rovnice (\ref{eq:eq1})
nebo Definice \ref{def:def1} na straně \pageref{def:def1}). Rovněž si vyzkoušíme používání odkazů \verb \ref  a \verb \pageref .

Na titulní straně je využito sázení nadpisu podle optického středu s využitím zlatého řezu. Tento postup byl
probírán na přednášce. Dále je použito odřádkování se
zadanou relativní velikostí 0.4em a 0.3em.
\section{Matematický text}
Nejprve se podíváme na sázení matematických symbolů a~výrazů v plynulém textu včetně sazby definic a vět s využitím balíku \texttt{amsthm}. Rovněž použijeme poznámku pod čarou s použitím příkazu \verb \footnote . Někdy je vhodné použít konstrukci \verb ${}$ , která říká, že matematický text nemá být zalomen.
\begin{def1}
\label{def:def1}
\textup{Turingův stroj} (TS) je definován jako šestice tvaru ${M = (Q, \Sigma, \Gamma, \delta, q_0, q_F)}$, kde:
\begin{itemize}
    \item Q je konečná množina \textup{vnitřních (řídicích) stavů,}
    \item ${\Sigma}$ je konečná množina symbolů nazývaná \textup{vstupní abeceda,} $\Delta \notin \Sigma$,
    \item $\Gamma$ je konečná množina symbolů, $\Sigma \subset \Gamma, \Delta \in \Gamma $, nazývaná             \textup{pásková abeceda}
    \item $\delta : (Q \setminus\{q_F\})\times \Gamma \rightarrow Q \times (\Gamma \cup \{L, R\})$, kde $L, R     \notin \Gamma$ je parciální \textup{přechodová funkce,}
    \item $q_0$ je \textup{počáteční stav,} $q_0 \in Q$ a
    \item $q_F$ je \textup{koncový stav,} $q_F \in Q$.
\end{itemize}
\end{def1}
Symbol $\Delta$ značí tzv. \emph{blank} (prázdný symbol), který se vyskytuje na místech pásky, která nebyla ještě použita (může ale být na pásku zapsán i později).

\emph{Konfigurace pásky} se skládá z nekonečného řetězce, který reprezentuje obsah pásky a pozice hlavy na tomto řetězci. Jedná se o prvek množiny $\{\gamma \delta^\omega | \gamma \in \Gamma^*\}\times \mathbb{N}$\footnote{Pro libovolnou abecedu $\Sigma$ je $\Sigma^\omega$ množina všech \emph{nekonečných} řetězců nad $\Sigma$, tj. nekonečných posloupností symbolů ze $\Sigma$. Pro připomenutí: $\Sigma^*$ je množina všech konečných řetězců nad $\Sigma$.}
\emph{Konfiguraci pásky} obvykle zapisujeme jako $\Delta xyz\underline{z}x\Delta\dots$ (podtržení značí pozici hlavy). \emph{Konfigurace stroje} je pak dána stavem řízení a konfigurací pásky. Formálně se jedná
o prvek množiny $Q \times \{\gamma \delta^\omega | \gamma \in \Gamma^*\}\times \mathbb{N}$.

\subsection{Podsekce obsahující větu a odkaz}
\begin{def1}
\label{def:def2}
\textup{Řetězec} $w$ \textup{nad abecedou} $\Sigma$ \textup{je přijat TS }$M$ jestliže M při aktivaci z počáteční konfigurace pásky $\underline{\Delta}w\Delta\ldots$ a počátečního stavu $q_0$ zastaví přechodem do koncového stavu $q_F$, tj. $(q_0,\Delta w, \Delta^w, 0) \displaystyle\vdashop\displaylimits_{M}^{*} (q_F, \gamma, n)$pro nějaké $\gamma \in \Gamma^*$ a $n \in \mathbb{N}$. 

Množinu $L(M) = \{w|w $ je přijat TS $M\}\subseteq \Sigma^*$ nazýváme \textup{jazyk přijímaný TS} $M$.
\end{def1}
Nyní si vyzkoušíme sazbu vět a důkazů opět s použitím balíku \texttt{amsthm}.
\begin{th1}
Třída jazyků, které jsou přijímány TS, odpovídá \textup{rekurzivně vyčíslitelným jazykům.}
\end{th1}
\begin{proof}
V důkaze vyjdeme z Definice \ref{def:def1} a \ref{def:def2}.
\end{proof}

\section{Rovnice a odkazy}
Složitější matematické formulace sázíme mimo plynulý text. Lze umístit několik výrazů na jeden řádek, ale pak je třeba tyto vhodně oddělit, například příkazem \verb \quad .

\begin{quote}
    $\sqrt[i]{x^3_i}$ \quad kde $x_i$ je $i$-té sudé číslo \quad $y^{2\cdot y_i}_i \neq y_i^{y_i^{y_i}}$
\end{quote}

V rovnici (\ref{eq:eq1}) jsou využity tři typy závorek s různou explicitně definovanou velikostí.

\begin{equation}
\label{eq:eq1}
    x = \bigg\{\Big( [a+b]*c\Big)^d \oplus 1 \bigg\}
\end{equation}

\begin{equation*}
    y = \lim_{x \to \infty} \frac{\sin^2 x + \cos^2 x}{\frac{1}{\log_{10} x}}
\end{equation*}

V této větě vidíme, jak vypadá implicitní vysázení limity $\lim_{n \to \infty} f(n)$ v normálním odstavci textu. Podobně je to i s dalšími symboly jako $\sum_{i=1}^n 2^i$ či $\bigcup_{A \in \mathcal{B}}A$. V~případě vzorců $\lim\limits_{n \to \infty} f(n)$ a $\sum\limits_{i=1}^n 2^i$ jsme si vynutili méně úspornou sazbu příkazem \verb \limits .
\begin{align}
    \int\limits_a^b f(x)\,dx &= -\int_b^a g(x)\,dx\\
    \overline{\overline{A \vee B}} &\Leftrightarrow \overline{\overline{A}\wedge \overline{B}}
\end{align}

\section{Matice}
Pro sázení matic se velmi často používá prostředí array a závorky (\verb \left , \verb \right ).
\clearpage
\begin{equation*}
\left(
\begin{array}{c c c}
    a+b& \widehat{\xi + \omega}& \hat{\pi}\\
    \vec{a} & \overset{\longleftrightarrow}{AC}& \beta
\end{array}
\right)
= 1 \iff \mathbb{Q} = \mathbb{R}
\end{equation*}

\begin{equation*}
    A = 
\left|\left| \begin{array}{c c c c}
     a_{11} & a_{12} & \dots & a_{1n}  \\
     a_{21} & a_{22} & \dots & a_{2n}  \\
     \vdots & \vdots & \ddots& \vdots  \\
     a_{m1} & a_{m2} & \dots & a_{mn}
\end{array}
\right| \right|
    = 
\left|
    \begin{array}{c c}
        t & u \\
        v & w
    \end{array}
\right|
    =  tw - uv
\end{equation*}

Prostředí \texttt{array} lze úspěšně využít i jinde.
\begin{equation*}
    \binom{n}{k} = \left\{
    \begin{array}{l l}
        \frac{n!}{k!(n-k)!} & \textstyle{\textup{pro }} 0\leq k \leq n  \\
        0                   & \textstyle{\textup{pro }} k < 0 \textstyle{\textup{nebo }} k > n
    \end{array}
\right.
\end{equation*}

\section{Závěrem}
V případě, že budete potřebovat vyjádřit matematickou konstrukci nebo symbol a nebude se Vám dařit jej nalézt v samotném \LaTeX u, doporučuji prostudovat možnosti balíku maker \AmS -\LaTeX.
\end{document}
